% Formulation - Rigorous existence proof and question construct

It was very helpful at first glance to see the formulation of this Sharpe ratio problem, as it was posed by a prior Cooper Union student Jorge Aguerrevere (ChE '12)\cite{back1}.  In getting this problem formulated there were many vector form substitutions we had to rationalize for ourselves and this pushed our understanding of the problem to new heights.  To bring the formulation to light, let us start with the Sharpe ratio and begin defining where some of the optimization variables and parameters might live.  Recall (\ref{eqS}) from page \pageref{eqS} of the report:
\begin{align*}
S &= \dfrac{E(R_p - R_f)}{\sigma_p}  =  \dfrac{E(R_p)}{\sigma_p}
\end{align*}

We have already said that $R_f$ is set to zero.  We also know that the Sharpe ratio being optimized, should be a scalar, and this means that both $\sigma_p$ and $E(R_p)$ are scalar quantities.  We will now create the vector quantities, and form a new problem to maximize the value of $S$:

\begin{equation}\label{eqProb1}
\text{Problem 1} = {{\mbox{max}}\atop{x\epsilon\Omega}} ~ \dfrac{E(R_p)}{\sigma_p}
\end{equation}

Such that: \indent\indent  $\Omega = \{ x\epsilon\Re : {1}^\top{x} = 1 , 0 \le x \le 1 \} $ \\[0.25cm]

$\Omega$ represents the compact set for this problem on which there are linear inequality and equality constraints on $x$.  The n-column vector $x$ represents the proportion of portfolio investment to go to each of $n$ stock choices.  The matrix ${1}$ is a constant ones column vector and lives in $\Re^{n\times 1}$.  ${1}^\top{x}$ is equivalent to the verbal statement that the sum of all the investment proportions must total to $1$.  The inequality constraints on $x$ tell us that all the elements of the vector are non-negative, and that all the elements of the n-column vector of $x$ are bounded above by $1$.

\subsubsection{Problem Set Up}
The expected portfolio return is the linear combination of the vector ${C}^\top$ with the n-column vector $x$, where we have defined the new vector $C$ to represent the total return of each individual stock over the entire time series being considered.  $C\epsilon\Re^{n\times n}$.  The daily returns from each individual stock are constructed by taking the day to day price change and dividing it by the current days price.  The daily return matrix $R\epsilon\Re^{m\times n}$.  More clarification on the design of the matrices $C$ and $R$, can be found in the Appendix Section 2. 

\begin{equation}
\label{eqTotPortReturn}
\text{Total Return} = {{C}^\top {x}} = {E(R_p)} 
\end{equation}
\begin{equation}
\label{eqDailyReturn}
\text{Daily Return}: {\Re^{n\times1}}\to{\Re^{m\times1}} ~~ \text{defined by }{{R} {x}}
\end{equation}
\begin{equation}
\label{eqResiduals}
\text{Variance} = \sum_{i=1}^m (Residuals)^2
\end{equation}
\begin{equation}
\label{eqVariance}
\text{Variance} = \sum_{i=1}^m ({{R}{x}} - {{C}^\top {x}})^2
\end{equation}

Apply an equivalent expression:
\begin{equation}
\label{eqVarianceNew}
\text{Variance} = \sum_{i=1}^m ({R}{x})^2 - {{x}^\top{C}}{{C}^\top {x}}
\end{equation}

\begin{equation}
\label{eqVar}
\text{Variance}= m {{x}^\top{R}^\top{R}{x}} - {{x}^\top{C}}{{C}^\top {x}}
\end{equation}



After defining the parameter matrices $R$ and $C$, and how they relate to the total return of the portfolio and variance of the portfolio we were able to set up the final optimization problem.  

\begin{equation}\label{eqProb1V}
\text{Problem 1} = {{\mbox{max}}\atop{x\epsilon\Omega}} ~ \dfrac{{C}^\top{x}}{\sqrt{m {x}^\top{R}^\top{R}{x} - {x}^\top{C}{C}^\top{x}}}
\end{equation} \\[-0.6cm]

\noindent Where:

$R$ =  m x n dimension matrix with elements corresponding to the daily returns from a \indent stock j, from day i to i+1;  i = 1 ... m and j = 1 ... n

$C$ = n-column vector with every element corresponding to the sum of return for a 
\indent given stock j over the given time series;  j = 1 ... n

$x$ = n-column vector for the proportion allocation of every stock j to be included in the
\indent portfolio;  j = 1 ... n

$m$ = number of days in the time series, time horizon

\noindent (See Appendix Section 2 for information on generating the parameter matrices, $R$ and $C$)

In the beginning of this formulation section, constraints were chosen that bounded the variable $x$ creating a compact set.  These constraints on the variable $x$ still hold.

In optimization routines it is common to minimize the objective function over the constrained set.  In addition, all of our techniques are developed with respect to minimization.  Taking the minimum of our Problem 1 argument multiplied by negative one, is equivalent to taking the maximum of our original Problem 1 argument.  The equivalent form of the problem stated in \ref{eqProb1V} follows:

\begin{equation}\label{eqP1}
\text{Problem 1} = {{\mbox{min}}\atop{x\epsilon\Omega}} ~ \dfrac{-{C}^\top{x}}{\sqrt{m {x}^\top{R}^\top{R}{x} - {x}^\top{C}{C}^\top{x}}}
\end{equation} \\[-0.6cm]

\underline{Quick note on shorting:}  The constraint on the decision variables puts into effect an upper and lower bound. Using 0 as a lower bound simply limits the problem to the situation where money is being spent on stocks, and no stocks are being shorted (shorting a stock means borrowing a stock and selling it, expecting that the stock's value will drop and it will be available for purchase at a lower price in the future). Shorting will not be considered in the portfolios in this project, because the risks involved in shorting stocks are very different to those involved in purchasing stocks. A portfolio that only has purchased stocks, in the worst-case scenario will lose its value completely. The value of the stock is bounded below. A portfolio that involves shorted stocks can leave the investor in debt, because there is no upper bound on the stock's value and, thus a shorted stock can result in a loss larger than the initial investment.\cite{back1}


\subsubsection{Problem Analysis}

This portfolio optimization problem utilizing Sharpe's ratio, is a non-linear programming problem subject to linear constraints on the variable $x$.  The objective function of this problem is non-linear, because just as with linear regression, the denominator of this function aims to sum together the residuals about a constant sloped line.  The function uses variance to assess risk in the portfolio allocation, and variance is inherently non-linear.  Variance is the square of all the residuals and must be impartial to both the positive and negative half-spaces.  The non-linearity stemming from the variance calculation really slows down the fmincon algorithm and we are looking into using the quadprog MATLAB function to speed up computational time.  

	The objective function has the variable $x$ in the numerator, and the square root of this variable $x$ in the denominator.  The objective function handles both the assessment of risk and the assessment of return.  This is a very nice property of optimizing with respect to Sharpe's ratio.  The constraints in this question are only used to ensure that the set is compact.  The set over which the objective function acts must be convex because the $Rx$ matrix, representing daily return, when squared must be a greater positive quantity than the ${C}^\top x$ matrix squared.  The validity in this statement is confirmed by the Cauchy-Schwartz inequality, which says that the square of the absolute value of a sum must be less than the sum of the square of absolute values.  $C$ is the sum of values in the $R$ matrix, and therefore the  ${x}^\top{C}{C}^\top{x}$ term must be less than the $m {x}^\top{R}^\top{R}{x}$ term.  The square root operator always acts on a positive quantity in the denominator of the objective function.  The objective function is defined for the linearly constrained variable vector $x$ and will always exist as the problem has been posed.  


\subsubsection{Method of Solving}\cite{back8}

MATLAB has a general purpose solver, that can handle both linear and non-linear programming problems.  We speak of the fmincon algorithm, which also accepts the linear constraints that we have posed.  The fmincon algorithm is great for beginners because it will assess what type of "scaled" algorithm to apply to a particular problem, based on the parameter matrices.  The fmincon algorithm has two large-scale algorithms, used for functions taking advantage a very sparse matrices.  An algorithm is large scale when it uses linear algebra that does not need to store, nor operate on, full matrices. This may be done internally by storing sparse matrices, and by using sparse linear algebra for computations whenever possible. MATLAB says to "Choose a medium-scale algorithm to access extra functionality, such as additional constraint types, or possibly for better performance".  A trust-region reflective algorithm may be implemented if the gradient of the function is accessible.  Interior-point algorithms in fmincon, quadprog, and linprog have many good characteristics, such as low memory usage and the ability to solve large problems quickly. However, their solutions can be slightly less accurate than those from other algorithms.  The answers to interior-point may be less accurate because the Interior point algorithm stays clear of the barrier functions, i.e. the boundary constraints.  Trust-region-reflective does not solve problems with the linear constraints we have specified, and as such MATLAB is implementing the Active-set algorithm, which is outdated and slower than most other models.  We will attempt to improve this clear inefficiency in the future, when necessary.
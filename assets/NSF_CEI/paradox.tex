%paradox
One of the central tenents of capitalism is that the actions of each business in an economic system working for personal gain benefit the economy as a whole. This is the notion that "greed is good"; each company fights to deliver the best services at the lowest cost in order to generate the highest profits for themselves, and only the most valuable and efficient survive. The size of the profits made by each company are based directly on the value they provide to the economy, and their employees divide these profits and spend them amongst other businesses in the free market in a feedback loop that seeks to continually reach an equilibrium of supply and demand that benefits all parties. Capitalism is inherently cutthroat, but ideally this danger becomes its greatest benefit.

The notion of a bank being TBTF represents a divergence from this idea; that the liquidation or fire sale acquisition of a bank operating in the free market that has reached the end of its life should not be tolerated. In these cases the government usually believes that the inevitable loss to depositors and other interconnected parties would be too great and too damaging to the economy as a whole. A pure capitalist would claim that the market correction following the total unassisted failure of a bank would eventually have the most beneficial effect on the economy. This argument tends to lose strength in considering of the potential loss of millions, billions, or trillions in bank assets during a fire sale and the following damage to growth funds, pension funds, and retirement funds that result from assets flooding the market as well as behavioral market selloff in the fear of recession.

During times of prosperity the government must publicly denounce, through public announcement or legislation, liquidity assistance for the most systemic financial firms.  Regulators and the government must promote a free and efficient market, discouraging the notion of "implicit backing" that can result in funding advantages, liquidity advantages, and greater mergers of size among the largest financial firms. 

Nevertheless, once crisis strikes regulators and the government increasingly manage to assist financial firms with liquidity while concurrently diffusing attention away from their actions.  A pure capitalist would argue that damping these crisis and correction periods results in a weaker economy in the long-term. It is interesting to note that regulator and government focus on short-term stability, has transcended to the management of the largest financial firms.  

Financial firms can incur heavy losses in a single night as a result of new and sensitive derivatives, high frequency trading, and leveraged hedging strategies, that require  principles of operation unthinkable in the past.  One of the growing principles of operation for these risky maneuvers by financial firms is the existence of a seemingly limitless short-term funding pool.  The leverage and hedging strategies are sustained through a heavy reliance on the short-term lending facilities, and the use of commercial paper to meet overnight reserve requirements.  These trading practices have resulted from the relentless pursuit of new cash flows that capitalism covets so greatly, making it difficult to criticize them from any standpoint other than the fact that they leave ordinary citizens increasingly exposed to their occasional massive losses.  In order to protect the average american, legislation aims to tighten the overnight reserve requirements.  \textbf{This may in effect make the short-term lending markets, even more important and systemic to the financial firms seeking to maintain capitalist profits.}  Congress had been warned about the systemic importance of short-term, interbank credit, after the 1984 bailout of Continental Illinois and the FDICIA.(CITATION)  Legislation will never be able to force profits from the mind of financiers, and the combination of leverage and short-term lending provides for a dangerous option in the minds of those with profitability decisions to make.  % More on this later when we say that in a build up to crisis the financial firms become increasingly dependent on short-term lending facilities
%
%(Is it a good idea to make banks more tuned into these short term funding markets)  (more were gonnna make fucking money profitability off the same asset power, so we will leverage alot more). 
  
The relationship between capitalism and the ordinary taxpayer is becoming strained as the profits of these new tools are increasingly held within the banks while their liabilities become more and more of a public issue. The government would prefer to remain a protector of the taxpayer when a bank takes a rare and exceptional misstep whose ramifications extend beyond its own clients, this role becomes difficult to fill when the conditions leading to such occasions have become more systemic than extraordinary.


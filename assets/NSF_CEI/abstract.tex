\thispagestyle{empty}
	
In this study, we examine the effects of increasingly leveraged Too Big To Fail institutions, and the effects this will have on the size of potential government bailouts.  We postulate that the effects of an inverted yield curve teamed with high leverage will form a volatile market.  The collapse of asset prices, and the scope of liquidity squeeze threaten to set off ever more viscous downward cycles for asset valuations.  

Deep and prolonged flattening yield curves for Treasury bills propagate into the municipal and corporate bond markets.  Knowing that rates in the securities market are dependent on treasury rats, we then move to examine the corporate profits of financial services that typically invest in the longer term securities market.  We attempt to show that hedging and leverage were implemented in greater force leading up to the financial crisis.  Leverage, hedging and arbitrage strategies, help the financial firms to pull artificial returns from already distorted asset values.

We postulate that the yield curve flattening is more dangerous today than ever before in history.  The flattening of the curve is correlated strongly with an increase in short-term funding and other indicators of hedging at play; such as the margin at broker/dealer indicator.  The shear scope of leveraging in asset purchases forces the government to step in with unprecedented liquidity for systemically important institutions.  In order to counteract the longer revaluation process, we postulate that the government/federal Reserve will have to provide larger and longer liquidation services to the financial arena.  In short, hedging and leverage in the financial world, has blown out of proportion and this cannot be reversed in fractional-reserve banking world; debt and leverage can always be obscured as was seen in asset-backed securities and collateralized debt obligations.  The short-term debit on margin does not seem to be fading away, and for that reason we believe that inherent danger remains in the leveraged banking world; fractional banking and leveraged time-sensitive investments.

We conclude our study with some important metrics to be watched carefully by the governing bodies.  TBTF is a continuing issue that the government is intimately required to facilitate, due to its responsibility to the individual wealth of citizens in this country.   Because the T-bill is a base rate on which all other asset value is derived, we examine additional indicators with respect to the inverting of the yield curve.  


%	
%	oo Big to Fail will become larger in scope, if the regulators and government do not step in to counteract asset bubbles due to leverage.
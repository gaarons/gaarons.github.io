% intro
\setcounter{page}{1}
%Govt. feels the need to protect the wealth of its people; citations of many occurrences of banking problems making there way to govt decisions

%; so goes the mantra of capitalism

Businesses must fail for others to succeed.  A failing business is not atypical and the United States economy is rarely disturbed by this process.  In fact, losses incurred by a fleeting business are resolved quite systematically under the Federal Bankruptcy Code (FBC).  The business and its counterparties are negatively effected by their shortcomings, but the externalities of the failure are insignificant.\cite{Kaufman}  The notion of Too Big to Fail (TBTF) results from businesses whose individual failure leads to disproportionately large adverse externalities that cannot be contained.\cite{Zhuo}  The shear size, asset value, and leveraging process that are inherent to financial firms (banks included) tend to bring about TBTF incidents.  The counterparties of financial firms are of an extraordinary quantity and their combined losses may have a profound impact on the United States economy.  

%\cite{Kaufman}
%In a TBTF regime loss allocation must be manipulated in order to protect in

In addition to the inherent TBTF problem in financial firms, there has been an across the board increase in leverage and more rampant use of short-term funding to cover margin requirements.  These practices have obscured true asset value and discouraged private sector buyouts in times of crisis.\cite{Kunt}   It has become increasingly difficult for a private buyer to value financial firm illiquid assets.  Typically a TBTF institution will be required to transfer assets from its failing business to a third party of impeccable solvency.  The third party will receive extremely discounted illiquid assets in exchange for their liquidity.  The discounted assets from the failing financial firm would otherwise have been sold in the open market with the risk of considerable fire sale losses.\cite{Kaufman}  If the third party is able to pass the necessary time without this liquid capital the illiquid assets, obtained from the failing firm, will provide extremely generous returns.  The TBTF business will survive having incurred serious losses, but the negative externalities of liquidation and excessive asset devaluation, traditionally seen in bankruptcy, will have been avoided.  The third party of impeccable solvency has ranged from wealthy individuals, to well positioned investment firms, and on a continued prevalence the federal government.  

The federal government is the lender of last resort and will only step into negotiations with the failing business if severe externalities would manifest from liquidation.  The federal government would always like these TBTF situations to be resolved in the private sector, where the illiquid assets of a failing company become the longer term investments of a successful business.\cite{Brewer}\cite{Dowd}  The federal government has been the prevailing third party, to purchase increasingly complex financial assets of uncertain value.  The federal government must either clear illiquid assets from a firm's balance sheet or inject liquidity, to reduce adverse externalities of an insolvent firm failure and liquidation.  The United States government is uniquely positioned as a constantly solvent third party in TBTF incidents.  The United States government has trouble with TBTF transfers of assets, because even the implicit guarantee of government support of troubled financial firms brings about moral hazard.\cite{Dowd}\cite{Zhuo}  The possibility of a creditor bailout creates moral hazard, no matter where the bailout funds originate, and it is moral hazard that provides the largest banks or other large financial firms with competitive funding advantages.\cite{Wallison}\cite{Strongin}  Moral hazard and the implicit government role as the third party buyer of assets, in TBTF incidents, significantly undermines market discipline.\cite{Rime} 

In the financial crisis of 2008, the United States government stepped into an historic third party TBTF position.\cite{Baker}   The benefactors of this decision should indeed be the entirety of consumers and businesses in the nation's economy.  In taking unprecedented steps to clear illiquid assets from the economy, the United States government and the Federal Reserve System, hoped to stabilize the financial system and all the services that are provided by that system.\cite{Dudley}  A major problem with TBTF financial institutions is that the financing services they offer may be disrupted and contribute to a longer economic suppression.  

In this report a closer examination is given to the prevalence of government, as well as federal reserve, in supporting TBTF financial institution.  The response of government to protect the wealth of its people and maintain economic stability is considered in the context of historical legislation on the part of United States regulators.








